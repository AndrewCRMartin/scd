\documentclass[a4paper]{article}

\newcommand{\sco}{{\tt scorecons}}
\title{Scoring conservation through diversity of observed residues}
\author{Andrew C.R.\ Martin}
\date{28th October 2022\\
Version~2 --- 29th October 2022}

\begin{document}

\maketitle

The problem is that we want to know how much diversity (or
conservation) there is at a given position in a sequence from a huge
set of data where most entries have the native amino acid. A simple
conservation score (e.g.\ from \sco) doesn't work very well because
the vast majority of the sequences being considered have the native
residue and, therefore, it is always scored as highly conserved. I
have already modified my implementation of \sco\ to allow this count
to be capped, or for logs to be taken of the numbers of observations
--- both of these help considerably.

An alternative is just to look at the observed residues that are
present and not consider the counts. I have already calculated the
number of mutated residues possible for each native residue assuming a
single base change (and also assuming that the native residue might be
encoded by any one of the appropriate codons).

In a very simple way we could then calculate a conservation score as:
\begin{equation}
  C = 1 - D
\label{eqn:C1}
\end{equation}
where $D$ is the observed diversity, calculated as:
\begin{equation}
  D = \frac{o}{N_n}
\label{eqn:D1}
\end{equation}
where $o$ is the number of unique amino acids observed at a given
position ({\bfseries excluding} the native) and $N_n = |\mathbf M_n|$ is the
number of unique mutated amino acids in $\mathbf M_n$ which is the set of
possible mutant residues, starting from native amino acid $n$ (also
excluding the native). This would give a value of 1 if all possible
amino acids are seen and a value of 0 if no mutations are seen.

However this does not consider similarity between the amino
acids.  What we need is for conservative mutations to contribute less
to the diversity.

We can use normalized PAM scores\footnote{The method for doing this is
  described in the Valdar \sco\ paper, but it is used by my
  implementation of \sco\ and is present in the BiopLib data directory
  as {\tt pet91.mat}, though the diagonal here is 10, so all values
  would need to be divided by 10.} (where the diagonal of the matrix
is equal and maximal and set to $1$) to score the similarity.

Consequently we replace Equation~\ref{eqn:D1} by:
\begin{equation}
  D = \frac{s_o}{N_n}
  \label{eqn:D2}
\end{equation}
where $s_o$ is defined as:
\begin{equation}
  s_o = \sum_{m \in \mathbf M_o} (1 - P(n,m))
\end{equation}
where $n$ is the native residue, $m$ is a mutant residue, $\mathbf M_o$ is
the set of observed mutated residues for $n$ and $P(n,m)$ is the
normalized PAM score for a mutation between $n$ and $m$. Given that
$P(n,n)$ is, by definition, 1.0, this means that, if all observed
residues were native, then $s$ would be 0.0, $D$ would be 0.0 and $C$
would be 1.0. (In practice, since we only look at the \emph{mutant}
residues and ignore the native residue, $s$ would, in any case be $0.0$.)

However, Equation~\ref{eqn:D2} needs to be normalized with respect to
the native residue since the divisor ($N_n$) is simply the number of
possible residues and doesn't take into account the range of
similarities between the native residue and the ones to which it can
be mutated (i.e.\ the minimum normalized PAM score for mutants of this
residue).

Consequently we replace Equation~\ref{eqn:D2} by:
\begin{equation}
  D = \frac{s_o}{t_p}
  \label{eqn:D3}
\end{equation}
where $t_p$ is defined as:
\begin{equation}
  t_p = N_n \times \max_{m \in \mathbf M_n} (1 - P(n,m))
\end{equation}

\noindent Combining Equations~\ref{eqn:C1} and~\ref{eqn:D3}, and writing
them out in full:
\begin{equation}
  C = 1 - \frac{\sum_{m \in \mathbf M_o} (1 - P(n,m))}{N_n \times \max_{m \in \mathbf M_n} (1 - P(n,m))}
\end{equation}
where:
\begin{itemize}
\item $C$ is the calculated conservation score
\item $n$ is the native mutant amino acid
\item $m$ is a mutant amino acid
\item $\mathbf M_o$ is the set of distinct observed mutant amino acids
\item $P(n,m)$ is the normalized PAM matrix score for mutation $n$ to $m$
\item $N_n$ is $|\mathbf M_n|$, the number of mutant amino acids possible with native residue $n$
\item $\mathbf M_n$ is the set of possible mutant amino acids for native residue $n$
\end{itemize}


\end{document}
